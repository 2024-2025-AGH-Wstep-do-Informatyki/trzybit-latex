\section{Kamil Szkarłat}


Wyrażenie matematyczne - wzór na delte:
\[ \Delta = b^2 - 4ac\]


\noindent % Usuwa wcięcie w tekście
\\Zdjecie przedstawiajace baton czekoladowy (patrz Figure~\ref{fig:baton})


\begin{figure}[htbp]
    \centering
    \includegraphics[width=0.3\linewidth]{pictures/trzybit.jpg}
    \caption{Oto baton czekoladowy}
    \label{fig:baton}
\end{figure}


\noindent
\\Table~\ref{tab:strzelcy_laliga} przedstawia aktualna klasyfikacje krola strzelcow w LaLiga.

\begin{table}[htbp]
\begin{tabular}{|
>{\columncolor[HTML]{C0C0C0}}l |l|l|l|l|l|}
\hline
\textbf{Lp.} & \cellcolor[HTML]{C0C0C0}\textbf{Imie} & \cellcolor[HTML]{C0C0C0}\textbf{Nazwisko} & \cellcolor[HTML]{C0C0C0}\textbf{Klub} & \cellcolor[HTML]{C0C0C0}\textbf{Bramki} & \cellcolor[HTML]{C0C0C0}\textbf{Asysty} \\ \hline
\textbf{1}   & Robert                                & Lewandowski                               & FC Barcelona                          & 12                                      & 2                                       \\ \hline
\textbf{2}   & Kylian                                & Mbappe                                    & Real Madryt                           & 6                                       & 1                                       \\ \hline
\textbf{3}   & Ayoze                                 & Perez                                     & Villarreal                            & 6                                       & 0                                       \\ \hline
\textbf{4}   & \multicolumn{1}{c|}{-}                & Raphinha                                  & FC Barcelona                          & 5                                       & 5                                       \\ \hline
\textbf{5}   & Vinicius                              & Junior                                    & Real Madryt                           & 5                                       & 4                                       \\ \hline
\end{tabular}
\label{tab:strzelcy_laliga}
\caption{Klasyfikacja króla strzelców LaLiga na stan październik 2024 r.}
\end{table}


\noindent
\\Przykładowa lista numerowana:
\begin{enumerate}
    \item Zakuć
    \item Zdać
    \item Zapomnieć
\end{enumerate}


\noindent
\\Przykładowa lista nienumerowana:
\begin{itemize}
    \item[-] Unia Europejska
    \item[-] NATO
    \item[-] BRICS
    \item[-] ONZ
\end{itemize}


\setlength{\parindent}{18pt}

\section*{Dlaczego warto studiować?}
Okres \textbf{studiów} to dla niektórych najlepszy czas w życiu. Wielu nie zdaje sobie sprawy jak wiele ich omija, gdy nie decyduja sie \textbf{studiować}.

Należy nadmienić co daje \textbf{studiowanie}. Przede wszystkim daje możliwość \underline{poznania wielu ludzi}, z którymi łaczyć nas może mnóstwo pasji i zainteresowań. Poza tym otrzymujemy dostep do ogromnej ilości \underline{materiałów i ksiazek} pomagajacych rozwijanie swoich umiejetności i wiedzy.

\noindent Dlatego warto \textbf{studiować} :) 

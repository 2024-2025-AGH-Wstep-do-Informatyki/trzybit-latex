\section{Jakub Szkaradek}


Wyrażenie matematyczne - wzór na pierwiastki funkcji kwadratowej:
\[ x_{0} = \frac{-b \pm \sqrt{\Delta}}{2a} \]


\noindent % brak wciecia
\\To jest samochód! (patrz Figure~\ref{fig:samochod})

\begin{figure}[htbp]
    \centering % srodkuje
    \includegraphics[width=0.5\linewidth]{pictures/samochód.jpg}
    \caption{A to jest samochód!}
    \label{fig:samochod}
\end{figure}


\noindent
\\Table~\ref{tab:mnozenie_do_3} przypomnienie mnozenia do 3

\begin{table}[hbtp]
\begin{tabular}{|
>{\columncolor[HTML]{FFCE93}}l |l|l|l|l}
\cline{1-4}
                 & \cellcolor[HTML]{FFCE93}\textbf{kolumna1} & \cellcolor[HTML]{FFCE93}\textbf{kolumna2} & \cellcolor[HTML]{FFCE93}\textbf{kolumna3} &  \\ \cline{1-4}
\textbf{wiersz1} & \textit{1}                                & \textit{2}                                & \textit{3}                                &  \\ \cline{1-4}
\textbf{wiersz2} & \textit{2}                                & \textit{4}                                & \textit{6}                                &  \\ \cline{1-4}
\textbf{wiersz3} & \textit{3}                                & \textit{6}                                & \textit{9}                                &  \\ \cline{1-4}
\end{tabular}
\label{tab:mnozenie_do_3}
\caption{test mnozenia}
\end{table}


\noindent
\\Test listy numerowanej
\begin{enumerate}
    \item pierwszy
    \item drugi
    \item trzeci
\end{enumerate}


\noindent
\\ Nie numerowany test!
\begin{itemize}
    \item[-] uno
    \item[-] dos
    \item[-] tres
    \item[-] quatro!
\end{itemize}


\setlength{\parindent}{18pt}

\section*{Studia}
Rozpoczynając studia, studenci często mierzą się z różnorodnymi wyzwaniami, które wymagają od nich nie tylko zaangażowania, ale również rozwijania nowych \textbf{umiejętności}. Przedmioty, które wydają się trudne, mogą wymagać intensywnej nauki i regularnych konsultacji z wykładowcami. Kluczową rolę odgrywa tutaj zarządzanie czasem, które pozwala na efektywne łączenie zajęć z życiem osobistym i innymi aktywnościami.

Wielu studentów angażuje się również w organizacje studenckie, co jest doskonałym sposobem na zdobycie cennego doświadczenia zawodowego jeszcze przed \underline{ukończeniem studiów}. Koła naukowe, konferencje oraz praktyki stanowią idealne miejsca, by budować sieć kontaktów zawodowych i zdobywać doświadczenie, które może okazać się kluczowe na rynku pracy.

\noindent \textbf{Kocham} Studiować! 